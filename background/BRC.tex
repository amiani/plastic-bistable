
\subsection*{Bistable Recurrent Cells}

Bistable Recurrent Cells (BRCs) \cite{vecoven2021brc} are another recurrent model similar to the GRU that allows for each individual unit of the memory to hold onto memories for an arbitrarily long time. The BRC modifies the update and reset gate equations to

\begin{gather*}
	\mathbf{z}_t = \sigma(\mathbf{w}_z \odot \mathbf{h}_{t-1} + W_z \mathbf{x}_t + \mathbf{b}_z)\\
	\mathbf{r}_t = 1 + \tanh(\mathbf{w}_r \odot \mathbf{h}_{t-1} + W_r \mathbf{x}_t + \mathbf{b_r}
\end{gather*}

where \(\mathbf{w}_z\) and \(\mathbf{w}_r\) are now weight vectors multiplied elementwise with the previous hidden state. The equation for \(\mathbf{\tilde{h}}_t\) is also modified, to

\[ \mathbf{\tilde{h}}_t = \tanh(\mathbf{r}_t \odot \mathbf{h}_{t-1} + W_h \mathbf{x}_t + \mathbf{b}_h) \]

Since these equations change all matrix multiplications with the hidden state vector to elementwise multiplications, all interactions between elements of the hidden state are removed. Instead, each unit of the hidden state interacts only with itself. Since it features only local computations, the BRC in this form is a much more plausible model of how biological neural networks might function than either RNNs or GRUs.




\subsection*{Neuromodulated Bistable Recurrent Cells}

\cite{vecoven2021brc} also introduced another form of the BRC that adds back in the interaction between elements of the hidden state in the equations for the update and reset gates. By relaxing the biological plausibility requirement, this modified BRC showed improved performance on a number of tasks. The update and reset gate equations are now

\begin{gather*}
	\mathbf{z}_t = \sigma(W_{zh} \mathbf{h}_{t-1} + W_{zx} \mathbf{x}_t + \mathbf{b}_z)\\
	\mathbf{r}_t = \sigma(W_{rh} \mathbf{h}_{t-1} + W_{rx} \mathbf{x}_t + \mathbf{b}_r)
\end{gather*}

whereas the equations for \(\mathbf{\tilde{h}}_t\) and the update to \(\mathbf{h}_t\) are the same as the standard BRC.