\subsection*{Bistable Recurrent Cells}

 Bistable Recurrent Cells (BRCs) \cite{vecoven2021brc} are another recurrent model similar in form to the GRU that allow for each individual unit of the memory vector to hold onto a value for an arbitrarily long time. The BRC features only local interaction of the hidden state, that is, that each hidden state neuron computes its activation value based only on the activation values of the neurons connected to it in the previous layer, the synaptic strengths of those connections and the activation value of the neuron itself at the previous timestep. Models that feature such local computations are interesting subjects of study because biological neural networks cannot do the global computations typically required for the operation of modern artificial neural networks. The BRC therefore modifies the update and reset gate equations to

\begin{gather*}
	\mathbf{z}_t = \sigma(\mathbf{w}_z \odot \mathbf{h}_{t-1} + W_z \mathbf{x}_t + \mathbf{b}_z)\\
	\mathbf{r}_t = 1 + \tanh(\mathbf{w}_r \odot \mathbf{h}_{t-1} + W_r \mathbf{x}_t + \mathbf{b_r}
\end{gather*}

where \(\mathbf{w}_z\) and \(\mathbf{w}_r\) are now weight vectors multiplied elementwise with the previous hidden state. The equation for \(\mathbf{\tilde{h}}_t\) is also modified, to

\[ \mathbf{\tilde{h}}_t = \tanh(\mathbf{r}_t \odot \mathbf{h}_{t-1} + W_h \mathbf{x}_t + \mathbf{b}_h) \]

with the update to \(\mathbf{h}_t\) being the same as the GRU. Since these equations change all matrix multiplications with the hidden state vector to elementwise multiplications, all interactions between elements of the hidden state are removed. Instead, each unit of the hidden state interacts only with itself and the input. Since it features only local computations, the BRC in this form is a much more plausible model of how biological neural networks might function than either RNNs or GRUs.




\subsection*{Neuromodulated Bistable Recurrent Cells}

Vecoven et al. \cite{vecoven2021brc} also introduced another form of the BRC that reintroduces the interaction between elements of the hidden state in the equations for the update and reset gates. By relaxing the local computation requirement, this modified BRC showed improved performance on a number of tasks. The update and reset gate equations are now

\begin{gather*}
	\mathbf{z}_t = \sigma(W_{zh} \mathbf{h}_{t-1} + W_{zx} \mathbf{x}_t + \mathbf{b}_z)\\
	\mathbf{r}_t = 1 + \tanh (W_{rh} \mathbf{h}_{t-1} + W_{rx} \mathbf{x}_t + \mathbf{b}_r)
\end{gather*}

whereas the equations for \(\mathbf{\tilde{h}}_t\) and the update to \(\mathbf{h}_t\) are the same as the standard BRC. The hidden state units are modulated by the activations of the input units and hidden state units, thus the name Neuromodulated Bistable Recurrent Cell (nBRC). This makes the nBRC very similar to the standard GRU, but crucially the equation for the candidate hidden state \(\mathbf{\tilde{h}}_t\) retains the elementwise multiplication of the reset gate and the previous hidden state. This difference is important because it allows the neurons of the nBRC (and BRC) to exhibit bistable dynamics\cite{vecoven2021brc}, which is crucial for their formation of long term memories.