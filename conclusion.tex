\section{Conclusion}

By combining the strengths of the differentiable plasticity and neuromodulated bistable recurrent cell methods, the plastic bistable recurrent cell introduced in this report is able to form and hold on to high dimensional memories. While this is also true of the nBRC method, the PBRC is able to do so with less training data, especially when it is equipped with a large memory vector.

An interesting further avenue of study might be the application of plasticity to the control paths of the nBRC, as opposed to the memory cells, as was explored here.